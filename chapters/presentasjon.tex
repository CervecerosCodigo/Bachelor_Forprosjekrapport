\chapter{Presentasjon}
\vspace{-1.5cm}
\begin{flushleft}
\renewcommand{\arraystretch}{1.5}
\begin{tabular}[ht]{@{}lp{100mm}@{}}
\textbf{Oppdragsgiver} & 
\begin{wrapfigure}{r}{0.25\textwidth}
\vspace{-0.75cm}
\hspace{-2.5cm}
\includegraphics[scale=0.15, keepaspectratio]{./img/presentasjon/mw_logo.png}
\end{wrapfigure}
\vspace{-0.25cm}
\href{http://www.makingwaves.no/}{\mw{}} (MW) \newline Kristian IVs gate 13 \newline 0164 Oslo \newline +47 22 206 020 \newline \href{mailto:post@makingwaves.com}{post@makingwaves.com} \\
\textbf{Prosjekteier} & 
\begin{wrapfigure}{r}{0.25\textwidth}
\vspace{-0.75cm}
\hspace{-2.5cm}
\includegraphics[scale=0.075, keepaspectratio]{./img/presentasjon/rb_logo.jpg}
\end{wrapfigure}
\vspace{-0.3cm}
\href{http://www.reddbarna.no/}{Redd Barna} \\
\textbf{Prosjekttittel} & Redd Barna \\ 
\textbf{Oppgave} & - \\ 
\textbf{Periode} & 04.01.2016 - 28.05.2016 \\ 
\textbf{Gruppenummer} & 19 \\ 
\textbf{Gruppemedlemmer} & Espen Bjorøy Zaal - s198599 \newline Lukas David Larsed - s198569 \newline Henrik Fischer Bjelland - s198570\newline Simen Flatby - s198577 \\ 
\textbf{Gruppeleder} & Espen Bjorøy Zaal \\ 
\textbf{Intern veileder} & Thor E. Hasle \\ 
\textbf{Eksterne veiledere} & Marius Slette Johansen \newline Manager for Portals, Commerce and Data Analysis \newline \href{mailto:marius.johansen@makingwaves.no}{marius.johansen@makingwaves.no} \newline +47 982 48 333 \newline \newline Kaare Øystein Trædal \newline Client Director \newline \href{mailto:kaare.tradal@makingwaves.com}{kaare.tradal@makingwaves.com} \newline +47 970 08 137 \\
\textbf{Prosjektside} & \url{http://cerveceroscodigo.org/} \\
\end{tabular} 
\end{flushleft}

\section{Oppdragsgiver}
\mw{} er eksperter på digital tjenesteutvikling og innovasjon. Hos \mw{} finner du rådgivning, design, teknologi, innholdsproduksjon og drift under samme tak. \mw{} er en digital innovasjonspartner for mange av Nordens største merkevarer og offentlige virksomheter.

\mw{} ønsker å inkludere studentene i sine prosjekter gjennom å gi en oppgave som er løsbar på normert tid, og som vil utfordre og lære studentene til å bruke sin kunnskap og lære mer om utvikling, prosjekt og teknologi.

Oppgaven er tenkt å løse en utfordring i et prosjekt hvor det er muligheter for å gjøre utvidelser eller nyutvikling av en tjeneste tilknyttet \rb{}.

\mw{} vil stille med faglig veiledning i prosjektet. Studentene kan forvente å jobbe sammen med \mw{} sine eksperter på .NET og søk.

\section{Oppgave}
Av informasjon studentene har fått tilgang til har det kommet frem at et av \rb{}'s største mål for 2016 er å skape bedre relasjoner og lojalitet til sine eksisterende givere. Dette målet har opphav i informasjon som sier at om en giver har vært medlem i 6 måneder i strekk er sjansen for at han forblir giver livet ut særdeles stor.

For å nå dette målet har gruppen sammen med \mw{} og \rb{} kommet frem til at det skal lages en personlig nettside eller en ``Min side". Oppgaven vil i korte trekk gå ut på å lage denne siden og alt som trengs for at den skal fungere som en selvstendig applikasjon.

Studentene vil få tilgang til en dump av det datasettet Redd Barna sitter på i form av medlemsregister, giverhistorikk og kundekommunikasjon. Denne dumpen kommer i CSV-format.

Om tiden skulle strekke til vil det også være aktuelt for gruppen å lage en administrasjonsside som ansatte i Redd Barna kan bruke til å se på statistikk.