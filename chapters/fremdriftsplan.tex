\chapter{Fremdriftsplan}

Fremdriftsplanen er kun tentativ og kommer sannsynlig å endres under prosjektets gang. Fremdriftsplanen er satt opp i form av \textit{fase} struktur med hensyn til preliminær planleging. Etter at prosjektet har startet kommer man til å overgå til smidig metodikk. 
\\

\begin{center}
\makebox[\textwidth][c]{  
\begin{tabulary}{1.2\textwidth}{|p{2cm}|p{2cm}|L|}
\hline
\textbf{Fase}	&	 \textbf{Periode}	&	\textbf{Beskrivelse}                                             \\	\hline

Oppstart 	& 	01.01.16 22.01.16	& 	
Oppstartfase der det er planlagt møter med veileder ved HiOA og MW. Innledende møter med kunde og brainstorming kring mulig løsning. Oppsett av prosjektverktøy, system for versjonskontroll, grupperoller og ansvarsområder. 
\\ \hline

Analyse og Planlegging		&	 25.01.16 12.02.16	&	Analyse av data og valg av teknologi. Møtevirksomhet med kunde (Redd Barna). Arkitektur og systemutvikling.
\\ \hline

Utvikling 1 &	15.02.16 12.03.16 	&	Deles inn i flere steg. Med testing mellom stegene. testing gjennomføres kontinuerlig.
\\ \hline

Utvikling 2	&	13.03.16 15.04.16 	&	Pågår frem til midten av april. Tilleggfunksjonalitet. Implementeringfase og integrasjon i eksisterende systemer.
\\ \hline

Påskeferie 	&	23.03.16 27.03.16	& Ferie. Fravær må <<påregnes>>. \\ \hline

Testing		&	16.04.16 01. 	&	Integrasjontesting og funksjonkontroll mot eksisterende systemer.
\\ \hline

Rapport		&	07.05.16 27.05.16 	&	Tid reservert kun til arbeid med rapporten. Øvrige deler av rapporten skrives kontinuerlig under prosjektets gang. \\ \hline

\end{tabulary}  
}
\end{center}
